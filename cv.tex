\documentclass[letterpaper,11pt,notitlepage]{article}
\usepackage{fancyhdr}
\usepackage[letterpaper,footskip=0.6in,left=1.in, right=1.in, top=1in, bottom=1in]{geometry}
%\usepackage[cm,empty]{fullpage}
%\usepackage[utf8]{inputenc}
\usepackage[T1]{fontenc}
\usepackage[english]{babel}
\usepackage{fontawesome5}
\usepackage{titlesec}
\usepackage{parskip}
\usepackage{pdfpages}
\usepackage{adjustbox}
\usepackage{enumitem}
\usepackage{bibentry}
\usepackage{natbib}
\usepackage{xhfill}
\usepackage{enumitem}
\usepackage{etoolbox}
\usepackage{doi}
\usepackage[level]{fmtcount}
\usepackage{longtable}

\usepackage{array}

\usepackage{xurl}
%\fontfamily{cmr}
\usepackage{marvosym} % symbols
\usepackage{academicons}
\usepackage{outlines}
%\usepackage{pdfsync}
\usepackage{xcolor,colortbl}
%\usepackage{ragged2e}
\usepackage{tabularx}
\usepackage{ifthen}
%\AtBeginDocument{\color{darkgray}}
\usepackage{textcomp}
\usepackage{orcidlink}  %% orcidID
%\usepackage{fontspec}

% rule command
\definecolor{mycolor}{HTML}{9f0035}
\newcommand{\myrule}{\textcolor{mycolor}{\rule{\linewidth}{4pt}}}
%% Orcid ID
\definecolor{orcidlogocol}{HTML}{A6CE39}

\fancyfoot[c]{\color[gray]{0.2}\footnotesize\copyright~Alberto Ruiz-Biestro}
\renewcommand{\headrulewidth}{0pt}
\fancyhead{}

\newcommand\blfootnote[1]{%
  \begingroup
  \renewcommand\thefootnote{}\footnote{#1}%
  \addtocounter{footnote}{-1}%
  \endgroup
}

\newcommand*{\webpage}{(see my \href{https://biestro.github.io}{website})}

\definecolor{infoboxcolor}{gray}{0.97}
\definecolor{myblue}{RGB}{0,0,202}
\definecolor{myred}{HTML}{d84878}

\usepackage{hyperref}
\hypersetup{
	pdftitle={CV Alberto Ruiz B},
	pdfauthor={Alberto Ruiz B},
	pdfsubject={CV from Alberto Ruiz Biestro},
	pdfpagemode=UseOutlines,
    hidelinks,
}

% Set spacing and stuff
\setlength{\parindent}{0pt}
\setlength{\parskip}{0.0em}

%\linespread{2.5}
%\renewcommand{\baselinestretch}{.82}
\setlist[itemize]{leftmargin=*,rightmargin=0pt,itemsep=2pt,topsep=2pt}


% Override fonts
%\usepackage[scaled=.94]{helvet}
\usepackage{helvet}
%\usepackage{times}
%\usepackage{newpxtext}
\renewcommand{\familydefault}{\sfdefault}

\newcommand{\hmargin}{1.7cm} % 1.8cm or more is ideal
\newcommand{\lcolsize}{.13\textwidth}
\newcommand{\rcolsize}{.82\textwidth}

\robustify{\bibentry}
\begin{document}
%\makeauthorbold{Alberto}
\pagestyle{fancy}
\begin{minipage}[t][\hmargin][c]{0.60\textwidth}
	
    {\Huge{Alberto Ruiz-Biestro} \orcidlink{0009-0005-4548-0081}}\\[0.8em]
	{\color{darkgray}\itshape\Large Curriculum Vitae}
	
\end{minipage}
\hfill
\begin{minipage}[t][\hmargin][c]{0.35\textwidth}
\raggedleft\color{darkgray}
Querétaro, México\\
\href{mailto:albertorbiestro@gmail.com}{albertorbiestro@gmail.com}\\
(+52) 448--116--1610\\
\href{https://biestro.github.io/}{biestro.github.io}\\
\end{minipage}

%\bigskip
\vspace{0.5in}


%%%%%%%%%%%%%%%%%%%%%%%%%%%%%%%%%%%%%%%%%%%%%%%%%%%%%%%%%%%
% \columncolor{myred}
\begin{longtable}{>{\raggedleft\arraybackslash}p{\lcolsize}>{\raggedright\arraybackslash}p{\rcolsize}}
\myrule{}    & \textcolor{mycolor}{\large\textsc{Education}}\\
2020 -- 2024 & \textbf{B.Sc. Engineering Physics}, \emph{Monterrey Institute of Technology}. GPA: 3.6\\
& \textbf{TOEFL iBT} Score: 108.\\[2.0em]
\myrule{}    & \textcolor{mycolor}{\large\textsc{Publications}}\\
1. & {\bf Alberto Ruiz-Biestro} and Julio C. Gutierrez-Vega. \\
   & ``Solutions of the Lippmann-Schwinger equation for confocal parabolic billiards''.\\
   & \emph{Phys. Rev. E.}, Mar 2024. \doi{10.1103/PhysRevE.109.034203}.\\[0.5em]
   & \textcolor{mycolor}{\textsc{Conference Presentations}}\\
2. &  {\bf  Mexican Optics and Photonics Meeting}. Poster presentation.\\
   & ``Lippmann-Schwinger equation in parabolic geometries''. Nov 2023.\\
1. & \textbf{National Space Activity Congress} (CONACES). ``Raman spectrometer design for biosignature detection'' (virtual). Nov 2021. \\[2em]
\myrule{}& \textcolor{mycolor}{\large\textsc{Awards}}\\
Apr 2023 & \textbf{Best Team Project}. \emph{International Centre for Theoretical Physics} \& \emph{Quantinuum}\\
         & Trieste, Italy. \ordinalnum{2} place in the quantum hackathon.\\[1em]
Aug 2020 & \textbf{Academic Merit Scholarship}, \emph{Monterrey Institute of Technology}.\\[2.0em]
\myrule{}& \textcolor{mycolor}{\large\textsc{Skills}}\\
\emph{Numerical} & Proficient in \textbf{Julia}, \textbf{\textsc{Matlab}}, \textbf{Python}, and \textbf{Linux}. Proven skills in Bash and \textbf{Git}.\\[0.5em]
\emph{Soft skills} & Analytical thinking, problem solving, collaboration, scientific communication.\\[2.0em]
\myrule{}& \textcolor{mycolor}{\large\textsc{Teaching Experience}}\\
Aug -- & {\bf Course assistant for Mathematical Methods for Physics}.\\
Dec 2023 & Graded homework and exams; held weekly advisory sessions.\\[0.5em]
Aug 2022 -- & {\bf Course assistant for Modern Electrodynamics}. \\
Jun 2023 & Graded homework and exams; held weekly advisory sessions.\\[2em]
\myrule{}& \textcolor{mycolor}{\large\textsc{Leadership}}\\
2022 - 2023 & \textbf{Quantum Computing Club} co-founder and VP.\\
\textbullet{}&  Organization of seminars, including one with \href{https://scholar.google.co.za/citations?user=JudpAs4AAAAJ&hl=en}{Dr. Benjamín Perez-García} on the implementation of Deutsch's algorithm with linear optics.\\
\textbullet{}& Organization and construction of a variety of courses that gave undergraduate students tools to program and analyze quantum algorithms.\\
\textbullet{}& Active participation in the organization of my institution's first \textbf{quantum hackathon}. Helped with dissemination and spreading the invitation to external faculty and students. \\[0.5em]
\textbullet{}& Coordinated and teaching of workshops in colaboration with the \emph{Physics Student Society} (AEF in Spanish) from Nuevo-Leon's Autonomous University (\href{https://www.uanl.mx/en/}{UANL}).\\[0.5em]
\textbullet{}& Organization, planning, and direction of quantum computing bootcamps, offering intensive courses to students from ITESM as well as from other universities. \\
\textbullet{}& Our outreach has grown beyond the state of Nuevo León.\\[1.0em]
2023 &  SPIE Student Chapter President\\
2022 -- 2023 & Given talks and short courses on Julia, Python, and \LaTeX.\\[2em]
\myrule{}& \textcolor{mycolor}{\large\textsc{Research Experience}}\\
Sep 2023 -- & \textbf{Photonics and Mathematical Optics Group}, \emph{Monterrey Institute of Technology}\\
Present & \emph{Advisors}: \href{https://scholar.google.com/citations?user=SXtXBWkAAAAJ&hl=en}{Julio C. Gutierrez-Vega}\\ 
& Implemented a Boundary Integral Method for solving the Lippmann-Schwinger (scattering) Equation.\\
& Development of meshes for discretization and parallel computation.\\
& Advanced theoretical methods and mathematical formulations for analytic results.\\[1em]

Apr 2023 & \textbf{International Centre for Theoretical Physics \& Quantinuum}. \emph{Trieste, Italy}\\
& \emph{Advisor}: Nathan Fitzpatrick \orcidlink{0000-0001-5819-9129} (\emph{Quantinuum})\\
\textbullet{}& Generated ground and excited state curves using a Quantum Krylov-subspace method along a reaction coordinate for an H\textsubscript{2} molecular Hamiltonian.\\
\textbullet{}& Development of hybrid quantum-classical algorithms with TKET and the InQuanto quantum chemistry platform; aided team in setting up and using \textbf{Git} for version control.\\
\textbullet{}& Collaborated with graduate students from diverse backgrounds. Our team received the \emph{Best Team Project} award, along with second place.\\[1em]
            &\\

Sep 2023 -- & \textbf{Photonics and Mathematical Optics Group}, \emph{Monterrey Institute of Technology}\\
Present & \emph{Advisors}: Dr. Antonio Ortiz-Ambriz \orcidlink{0000-0002-8302-0861} Dr. Gerardo Fox \orcidlink{0000-0002-7696-0767} Dr. Servando López \orcidlink{0000-0002-8492-1709}\\
\textbullet{}& Numerical simulation of the \emph{Nonlinear Schrodinger Equation} through \emph{pseudo-spectral method} (split-step Fourier) and numerical solutions of Boundary Value Problems (shooting method, finite differences, etc.).\\
\textbullet{}& Developed audio-identification algorithm in order to identify an audio recording from a microphone (FFT and signal-processing methods).\\
\textbullet{}& Analyzed the travelling-salesman-problem through simulated annealing; simulated the dynamics and critical points of the Lenz-Ising model.\\
\textbullet{}& Developed \textbf{Genetic algorithms} and \textbf{Neural Networks}; Experience with \textbf{Agent Based Modeling}.
\end{longtable}
\end{document}
